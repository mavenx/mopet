\chapter{Introduction}

You know the problem: You are on a sightseeing-trep in a city and want to take a picture of some
building or just a nice place. But there are annoying persons moving around in front
of the object!

So why don't you use our MOPET Software?
All you need is a series of about 10 pictures of the object. The software 
detects all moving persons and objects and removes them.



\chapter{Methods}

Unfortunately, we did not find any papers or other literature to this topic. Most of the methods for estimating
background are using a video stream as input and estimate the optical flow.
We decided to use a series of images,
because the quality of a single picture is much better than the quality of a frame in a video.

In a series of images it is very difficult to estimate the optical flow, because the time between
two images is very much longer, than in a video stream.

Therefore, we came up with our own idea.

\section{Registration of images}

We don't assume that the series of pictures are taken using a tripod.
Because of small movements of the camera during a continuous shooting, the image section may move
around.
Therefore, we need to align the images with the following steps.

\begin{itemize}
 \item Applying a feature-detector to all images and calculating feature descriptors.
  As a feature-detector we used FAST and for the descriptor SURF.
 \item The feature-descriptors of all images are matched to the features of the first image.
 \item RANSAC is used to estimate a homography, which is robust against outliers due to moving foreground.
 \item All images are then warped, to the position of the first image.
\end{itemize}



\chapter{Implementation}

We implemented the algorithm in C++ with OpenCV.

\chapter{Evaluation / Results}







